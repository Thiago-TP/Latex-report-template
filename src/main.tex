\documentclass[a4paper, 11pt]{article}  % especificação do documento com opções padrão
  \usepackage[T1]{fontenc}    % suporte para letras com acento e hifenização 
                            % (https://tex.stackexchange.com/questions/664/why-should-i-use-usepackaget1fontenc)
                            
% definição da cor tema (mais cores em latexcolor.com)
\usepackage{tikz}
    \definecolor{themecolor}{rgb}{1.0, 0.65, 0.0}
    
\usepackage[margin=2.5cm]{geometry} % margem do texto uniforme e igual a 2.5cm
\usepackage[brazil]{babel}          % troca 'Summary' por 'Sumário', 'Figure' por 'Figura', 'Table' por 'Tabela', etc.
\usepackage{charter}                % coloca a fonte do relatório como charter
\usepackage{helvet}                 % estiliza a fonte sans serif 
                                    % (https://linorg.usp.br/CTAN/macros/latex/required/psnfss/psnfss2e.pdf)
\usepackage{multirow}               % fornece o comando \multirow{}{}{} para mesclar células numa mesma linha da tabela
\usepackage{booktabs}               % disponibiliza comandos que deixam a tabela mais bonita 
                                    % (https://linorg.usp.br/CTAN/macros/latex/contrib/booktabs/booktabs.pdf) 
\usepackage{graphicx, float}        % pacotes para inserção de figuras
    \graphicspath{{figures/}}       % pasta figures se torna o lugar de procura padrão do \includegraphics
\usepackage{amsmath}                % pacote básico para suporte a equações e matemática em geral 
                                    % (https://linorg.usp.br/CTAN/macros/latex/required/amsmath/amsldoc.pdf)
\input{setup_listings.tex}          % formatação de códigos python e matlab
                                    % (http://users.ece.utexas.edu/~garg/dist/listings.pdf)

% estiliza texto das legendas
\usepackage{caption, subcaption}
    \captionsetup{
        labelfont={sf, bf},         % 'Figura X:' em negrito e sans serif
        format={hang},              % quebra de linha no texto da legenda é identado
        textfont=it                 % texto da legenda em itálico
    }

% estilização dos títulos das seções, subseções, e subsubseções
\usepackage{titlesec} 
    \titleformat*{\section}{\large\bfseries\sffamily}   % fonte grande, negrito, sans serif
    \titleformat*{\subsection}{\bfseries\sffamily}      % negrito, sans serif
    \titleformat*{\subsubsection}{\sffamily}            % sans serif

% escreve informações que devem aparecer no topo e base da página
\usepackage{fancyhdr, lastpage}
    \setlength{\headheight}{15pt}                       % espaçamento de segurança entre o cabeçalho e o texto
    \fancyhead{}
    \fancyhead[r]{Código do curso -- Prof. Dr. Nome do professor}
    \fancyhead[l]{\rightmark}                           % subseção atual
    
    \fancyfoot{} 
    \fancyfoot[r]{\thepage \ de \pageref*{LastPage}}    % página X de Y
    \fancyfoot[l]{\leftmark}                            % seção atual
    
    \renewcommand{\headrulewidth}{0.00mm}               % sem linha no topo da página (tire do 0 para criá-la)
    \renewcommand{\footrulewidth}{0.25mm}               % com linha no fundo da página (ponha em 0 para tirá-la)
    \pagestyle{fancy}                                   % moldura acima = padrão do documento (exceto página do título)

% coloração dos links
\usepackage{hyperref}
    \hypersetup{
        colorlinks,
        linkcolor   = themecolor!80!black,  % mistura a cor tema com um pouco de preto
        urlcolor    = themecolor!80!black
    }


% comando para a criação da figura no título
\newcommand{\cover}[1]{ 
    \begin{center}
        \includegraphics[width=5cm]{#1}
    \end{center}
}

% comando para destacar texto (caixa colorida)
\newcommand{\highlightbox}[1]{\noindent
    \colorbox{themecolor!25}{
        \begin{minipage}{\linewidth}
            #1
        \end{minipage}
    }
}
                 % aciona o código que prepara todo o documento

  % título, autores e data
  \title{\textbf{\sffamily     % deixa em negrito + sans serif
        Turma X, Grupo XX\\
        Relatório do trabalho de XX:\\
        Nome do trabalho
    }
  }
  \author{
    Autor 1
    \thanks{Matrícula 1 -- email 1}     % se desnecessário tire os \thanks
    \and
    Autor 2 
    \thanks{Matrícula 2 -- email 2}
    \and
    Autor 3 
    \thanks{Matrícula 3 -- email 3}
    % \and
    % Autor 4 
    % \thanks{Matrícula 4 -- email 4}   % caso hajam mais autores, simplesmente repita o \and
  }
  \date{
    Universidade de Brasília, XX de YY de 20ZZ
  }

% início do pdf  
\begin{document}
    \maketitle  % imprime título, autores, data
    \abstract{  % resumo
        O \LaTeX~é uma poderosa ferramenta de formatação de texto voltada para artigos acadêmicos.
        Embora demore a ser dominada, a excelente apresentação providenciada pela ferramenta é uma vantagem que deve ser considerada.
        Este repositório procura suprir a necessidade do usuário por um \textit{template} bem feito de relatório em \LaTeX~sem, entretanto, exigir conhecimentos profundos da ferramenta.
        Ao longo deste texto a estrutura do repositório, assim como as funcionalidades específicas deste projeto, são explicadas. 
        Abaixo mostra-se uma bela imagem opcional de capa.
    }\\
    
    \cover{capa}                        % imagem de capa (retire o comando se desnecessário)
    \tableofcontents                    % imprime sumário
    \thispagestyle{empty}               % tira o número da página

    \clearpage                          % nova página
    
    \section{Introdução}
    Para organizar o seu relatório em \LaTeX, recomenda-se usar a organização utilizada neste repositório:
    \begin{itemize}
        \item uma pasta {\tt figures} com todas as figuras, gráficos e etc. do trabalho;
        \item uma pasta {\tt listings} com todas os códigos a serem incluídos no relatório;
        \item uma pasta {\tt sections} com todas as seções do relatório (arquivos {\tt.tex});
        \item uma pasta {\tt src} com os arquivos {\tt.tex} que preparam e estruturam o relatório.
    \end{itemize}

    Dependendo do tamanho do documento, pode vir a ser confortável criar uma 5\textordfeminine~pasta que contenha subseções, a {\tt subsections}.
    Por outro lado, caso o documento seja livro ao invés de artigo, o diretório {\tt sections} pode ser substituído por {\tt chapters}, ou {\tt parts}. 

    A pasta {\tt src} confina a {\tt main.tex}, {\tt preâmbulo.tex} e o {\tt setup\_listings.tex}.
    É na {\tt main} que o tamanho da fonte, título, autores e data são definidos, assim como as seções são incluídas; 
    este é o único arquivo que deve ser compilado, gerando o {\tt.pdf} do relatório.
    A compilação deve ocorrer via pdfLaTeX. Caso esteja no Overleaf, o compilador do projeto é exposto no Menu $\to$ Settings $\to$ Compiler.
    
    No {\tt preâmbulo.tex}, importa-se as pacotes do \LaTeX~necessárias para formatar o {\tt.pdf} em si, como a fonte do texto, cor tema do relatório, estilização das legendas, e moldura da página (essa que mostra o nome do professor, número da página, código da disciplina e etc.).
    Certifique-se de possuir esses pacotes instalados antes de compilar a {\tt main}.

    O {\tt setup\_listings.tex} formata a exibição de códigos em Python e MATLAB de maneira minuciosa.\\

    Ao todo, não deve ser do interesse do leitor alterar os setups e o preâmbulo além do necessário para mudar a cor tema, moldura ou formatação dos códigos, mas sim escrever nos {\tt.tex} da {\tt sections}.
    Caso encontre alguma dúvida ou problema, contate o autor do template pelo próprio Github em X ou mande mensagem para {\tt thiagotomasdepaula@gmail.com}.     % introdução
    \section{Básico do \LaTeX}
\label{sec:básico}
    Apresentada a estrutura do projeto, passa-se a apresentar maneiras de escrever o {\tt.pdf} em si.
    Não serão detalhados os documentos de preâmbulo e setups visto o texto ficaria longo demais e os códigos estão bem comentados.
    No que segue, espera-se que o leitor veja o {\tt.pdf} e o código {\tt.tex} da seção correspondente em paralelo.

    \subsection{Citações automáticas}
    \label{sec:citações}
        O \LaTeX tem a extremamente útil capacidade de tornar automática a citação a figuras, equações matemáticas, tabelas, seções (e afins), e, é claro, referências bibliográficas, entre outros.
        Para tanto, simplesmente coloque o comando \verb|\label{}| dentro do ambiente ({\tt figure}, {\tt table}, {\tt equation}, etc.) ou logo abaixo do comando (\verb|\section|, etc.) cuja referência é necessária, e faça a citação através do comando \verb|\ref{}| ou \verb|\autoref{}|.
        Em particular, equações também podem ser referidas por \verb|\eqref{}|

        Por exemplo, foi colocado \verb|\label{sec:citações}| nesta subseção, 
        de forma que escrever \verb|\ref{sec:citações}| retorna o número \ref{sec:citações},
        enquanto que escrever \verb|\autoref{sec:citações}| retorna \autoref{sec:citações}.
        Confira alguns exemplos a seguir.
        \begin{itemize}
            \item ``\verb|A seção \ref{sec:básico}|'' $\to$ ``A seção \ref{sec:básico}''.
            \item ``\verb|A \autoref{sec:básico}|'' $\to$ ``A \autoref{sec:básico}''.
            \item ``\verb|A equação \ref{eq:seno}|'' $\to$ ``A equação \ref{eq:seno}''.
            \item ``\verb|A \autoref{eq:seno}|'' $\to$ ``A \autoref{eq:seno}''.
            \item ``\verb|A equação \eqref{eq:seno}|'' $\to$ ``A equação \eqref{eq:seno}''.
            \item ``\verb|A figura \ref{fig:exemplo}|'' $\to$ ``A figura \ref{fig:exemplo}''.
            \item ``\verb|A \autoref{fig:exemplo}|'' $\to$ ``A \autoref{fig:exemplo}''.
            \item ``\verb|A figura \ref{subfig:exemplo}|'' $\to$ ``A figura \ref{subfig:exemplo}''.
            \item ``\verb|A \autoref{subfig:exemplo}|'' $\to$ ``A \autoref{subfig:exemplo}''.
        \end{itemize}

        Por outro lado, referências bibliográficas são definidas e listadas na seção Referências, que não aparece no sumário.
        Para criar uma ficha de referências, escreva \verb|\bibitem{}| dentro do ambiente \verb|thebibliography|.
        O argumento do \verb|\bibitem{}| é um apelido da referência respectiva, e a referência em si (título, tipo, autores, etc.) deve ser explicitada abaixo desse comando.

        O código abaixo gera a bibliografia deste modelo. Note que este é um método manual, que pode vir a ser inconveniente para projetos com várias referências, mas em geral relatórios da graduação tendem a ter poucas fontes.
        O {\tt 5} é o número máximo de referências esperado, e é usado pelo \LaTeX~para garantir que a lista fique alinhada. 
        Confira mais informações de como fazer uma bibliografia 
        \href{https://www.overleaf.com/learn/latex/Bibliography_management_with_bibtex}{aqui}.

        \begin{verbatim}
\begin{thebibliography}{5}
    \bibitem{tipler} 
        Paul A. Tipler e Gene Mosca.
        \textit{Física Volume 2, 5\textordfeminine Edição}. 
        LTC--Livros Técnicos e Científicos Editora S.A., Rio de Janeiro, 2006. 
    
    \bibitem{taylor}
        John R. Taylor. 
        \textit{An Introduction to Error Analysis, Second Edition}. 
        University Science Books, Sausalito (CA), 1997. 

    \bibitem{britannica}
        Britannica, The Editors of Encyclopaedia. 
        "servomechanism".
        Encyclopedia Britannica, 14 May. 2013, 
        \url{https://www.britannica.com/technology/servomechanism}. 
        Acessado 15 de janeiro de 2023.
\end{thebibliography}
        \end{verbatim}

        Note o comando \verb|\url{}| na última referência: é uma forma de inserir url's clicáveis no \LaTeX.
        Esse tipo de referência, assim como hyperlinks, serão vistos mais à frente.
    
    \subsection{Inserção de equações}
        Uma das grandes vantagens do \LaTeX~é a facilidade em se inserir equações assim como a qualidade da formatação do texto matemático.
        Para colocar uma equação numerada, use o ambiente {\tt equation}. Para uma não numerada, use {\tt equation*} ou \verb|$$|.

        Por exemplo, o código
        \begin{verbatim}
        \begin{equation}
            \sin(z) = 
            z 
            - \frac{z^3}{3!} 
            + \frac{z^5}{5!}
            - \dots
            = 
            \sum_{n=0}^\infty \frac{z^{2n+1}}{(2n+1)!}
            \label{eq:seno}
        \end{equation}
        \end{verbatim}

        gera a \autoref{eq:seno},
        \begin{equation}
            \sin(z) = 
            z 
            - \frac{z^3}{3!} 
            + \frac{z^5}{5!}
            - \dots
            = 
            \sum_{n=0}^\infty \frac{z^{2n+1}}{(2n+1)!}
            \label{eq:seno}
        \end{equation}

        enquanto que o código 
        \begin{verbatim}
        $$ 
            \Gamma(z) =
            \int_0^\infty t^{z-1} e^{-t} \ dt
            =
            (z-1)!
        $$
        \end{verbatim}

        gera a equação abaixo.
        $$ 
            \Gamma(z) =
            \int_0^\infty t^{z-1} e^{-t} \ dt
            =
            (z-1)!
        $$

        Para colocar expressões matemáticas na linha do texto, use \verb|$|. 
        Mais detalhes \href{https://www.overleaf.com/learn/latex/Mathematical_expressions}{aqui}.

    \subsection{Negrito, itálico e sublinhado}
        Para deixar em itálico, negritar, ou sublinhar um texto, use os comandos
        \verb|\textit{}|, \verb|\textbf{}| e \verb|\underline{}|, respectivamente. 
        O {\tt preâmbulo.tex} fornece ainda o comando \verb|\highlightbox{}| para destacar blocos de texto como, por exemplo, citações longas.
        Confira um exemplo de uso de cada comando a  seguir.\\

        \highlightbox{\textit{
                ``...l'\textbf{amour} est cent fois meilleur que la \underline{haine}. 
                L'\textbf{espoir} est meilleur que la \underline{peur}.
                L'\textbf{optimisme} est meilleur que le \underline{désespoir}.''
            }
        }

        Para mais detalhes, procure
        \href{https://www.overleaf.com/learn/latex/Bold%2C_italics_and_underlining#Underlined_text}{aqui}.
        Por fim, repare que as aspas da citação foram feitas com acento agudo e apóstrofe, que é a forma padrrão do \LaTeX~para criá-las.

    \subsection{Atalhos de espaçamento vertical e quebra de linha}
        Um parágrafo novo pode ser iniciado deixando uma linha em branco entre dois blocos de texto.
        Para iniciar nova linha sem criar um parágrafo, use \verb|\\| ao final da linha.
        Por exemplo, colocar \verb|\\| aqui\\
        gera linha aqui.
        Usar o \verb|\\| além da linha em branco entre os blocos de texto gera tanto um parágrafo quanto um pequeno espaçamento vertical.
        
        Por exemplo, este é um primeiro parágrafo,\\

        E este é um segundo parágrafo com \verb|\\| e linha em branco.

    \subsection{Inserção de figuras}
        Para inserir uma figura no relatório use o comando \verb|\includegraphics{}| dentro do ambiente {\tt figure}.
        Tenha em mente que o arquivo da imagem deve estar em {\tt figures}.
        Por exemplo, o código
        \begin{verbatim}
        \begin{figure}[H]\centering
            \includegraphics[width=10cm]{gráfico do teste.pdf}
            \caption{Figura de teste.}
            \label{fig:exemplo}
        \end{figure}
        \end{verbatim}

        cria a imagem abaixo. O H serve para posicionar a figura no lugar mais próximo possível do esperado.
        \begin{figure}[H]\centering
            \includegraphics[width=10cm]{gráfico do teste.pdf}
            \caption{Figura de teste.}
            \label{fig:exemplo}
        \end{figure}

        Para incluir múltiplas figuras, use o ambiente {\tt subfigure} dentro do {\tt figure}.
        Por exemplo, o código
        \begin{verbatim}
        \begin{figure}[H]\centering
            \begin{subfigure}{.45\textwidth}\centering
                \includegraphics[width=3cm]{unb_logo.png}                
                \caption{Figura 1.}
                \label{subfig:exemplo}
            \end{subfigure}
            \hfill
            \begin{subfigure}{.45\textwidth}\centering
                \includegraphics[width=3cm]{unb_logo.png}                
                \caption{Figura 2.}
            \end{subfigure}
            \caption{Figura dupla.}
        \end{figure}
        \end{verbatim}

        gera a figura dupla baixo.
        \begin{figure}[H]\centering
            \begin{subfigure}{.45\textwidth}\centering
                \includegraphics[width=3cm]{unb_logo.png}                
                \caption{Figura 1.}
                \label{subfig:exemplo}
            \end{subfigure}
            \hfill
            \begin{subfigure}{.45\textwidth}\centering
                \includegraphics[width=3cm]{unb_logo.png}                
                \caption{Figura 2.}
            \end{subfigure}
            \caption{Figura dupla.}
        \end{figure}

        Note que o \verb|.45\textwidth| garante que as figuras podem ficar lado a lado com alguma folga. 
        No caso de 3 figuras lado a lado, a largura de cada uma não deve passar de \verb|.33\textwidth|.
        Por fim, veja mais detalhes sobre inserção de figuras 
        \href{https://www.overleaf.com/learn/latex/Inserting_Images}{aqui}.         % seção postiça 1
    \section{Conhecimentos mais avançados}
    \subsection{Inserção de código Python e MATLAB}
        É comum nas engenharias o ensino de Python, e para aquelas que lidam com sinais e sistemas,
        como elétrica e mecatrônica, o uso do Matlab.
        Um código pode ser inserido no texto através do ambiente {\tt lstlisting} fornecido pelo pacote {\tt listings};
        por exemplo, o código 
        \begin{verbatim}
        \begin{lstlisting}[language=Python]
        print('Hello World!')
        \end{lstlisting}
        \end{verbatim}

        gera
        \begin{lstlisting}[language=Python]
        print('Hello World!')
        \end{lstlisting}

        ao passo que 
        \begin{verbatim}
        \lstset{mystyle}
        \begin{lstlisting}[language=Matlab]
        clc; hold on; grid on;
        for a = 0:10
            G = tf(1, [1 a]);
            bode(G);
        end
        'done!'
        \end{lstlisting}
        \end{verbatim}

        gera
        \begin{lstlisting}[language=Matlab]
        clc; hold on; grid on;
        for a = 0:10
            G = tf(1, [1 a]);   % malha aberta com pólo em -a
            bode(G);
        end
        'done!'
        \end{lstlisting}

        Para trecchos maiores de código, recomenda-se usar o comando \verb|\lstinputlisting{}|, 
        que inclui um código no texto de maneira automática.
        Neste modelo, o comando espera que os programas estejam na pasta {\tt listings}.
        Por exemplo, o código
        \begin{verbatim}
        \lstinputlisting[
            language=Python, label={lst:código longo},
            caption={
                Código Python de maior tamanho.
            }]{exemplo.py}
        \end{verbatim}

        resulta no \autoref{lst:código longo}.
        \lstinputlisting[
            language=Python, label={lst:código longo},
            caption={
                Código Python de maior tamanho.
            }]{exemplo.py}

        Note que as palavras chave de cada linguagem ficam destacadas, facilitando a leitura.
        Caso queira mudar as cores, modifique o {\tt setup\_listings.tex} na {\tt main}.
        Para se aprofundar no uso do pacote {\tt listings}, veja 
        \href{https://www.overleaf.com/learn/latex/Code_listing}{este site} ou 
        \href{http://users.ece.utexas.edu/~garg/dist/listings.pdf}{a própria documentação}.


    \subsection{Hyperlinks e URLs}
        URLs clicáveis e hyperlinks podem ser inseridos no relatório através dos comandos
        \verb|\url{}| e \verb|\href{}| disponibilizados no pacote {\tt hyperref}.
        Recomenda-se que esse pacote seja o último a ser incluído para evitar conflitos com outros pacotes que mexem com citações,
        e também permite a customização das cores dos links, entre outros.

        Confira um exemplo de uso de cada comando abaixo.
        Para mais detalhes, vide 
        \href{https://www.overleaf.com/learn/latex/Hyperlinks}{este site} ou 
        \href{https://linorg.usp.br/CTAN/macros/latex/contrib/hyperref/doc/hyperref-doc.pdf}{a própria documentação}.
        \begin{itemize}
            \item 
            ``\verb|O site \url{https://latexcolor.com} é ótimo|'' $\to$
            ``O site \url{https://latexcolor.com} é ótimo''

            \item 
            ``\verb|Acesse documentações dos pacotes \href{https://ctan.org}{aqui}.|'' $\to$
            ``Acesse documentações dos pacotes \href{https://ctan.org}{aqui}.''
        \end{itemize}
    
    \subsection{Criação de tabela}
        Inserir uma tabela em \LaTeX~pode vir a ser relativamente complicado.
        Se a tabela não possui células que abrangem mais de uma coluna ou linha por vez,
        o código fica relativamente simples: 
        basta usar o ambiente {\tt tabular} dentro do ambiente {\tt table}, e usar os separadores de coluna e linha
        \verb|&| e \verb|\\| para preencher a tabela em si.
        Por exemplo, o código
        \begin{verbatim}
        \begin{table}[H]\centering
            \begin{tabular}{c | c c}
                \toprule
                $t_\text{exec}$ ($\mu$s)    & $C$       & $f$ (MHz) \\
                \midrule
                \midrule
                $0$                         & $0$       & $50$      \\
                $50$                        & $2500$    & $50$      \\
                $100$                       & $5000$    & $50$      \\
                \bottomrule
            \end{tabular}
            \caption{Tabela simples.}
            \label{tab:simples}
        \end{table}
        \end{verbatim}

        gera a \autoref{tab:simples}.
        \begin{table}[H]\centering
            \begin{tabular}{c | c c}
                \toprule
                $t_\text{exec}$ ($\mu$s)    & $C$       & $f$ (MHz) \\
                \midrule
                \midrule
                $0$                         & $0$       & $50$      \\
                $50$                        & $2500$    & $50$      \\
                $100$                       & $5000$    & $50$      \\
                \bottomrule
            \end{tabular}
            \caption{Tabela simples.}
            \label{tab:simples}
        \end{table}

        Como no caso de inserir uma figura, o {\tt H} serve para posicionar a tabela precisamente.
        Os argumentos {\tt c} e {\tt |} do tabular centraliza uma coluna e cria uma linha vertical entre colunas, respectivamente.
        A quantidade de {\tt c}'s estabelece a quantidade de colunas da tabela.
        Os comandos \verb|\toprule|, \verb|\midrule| e \verb|\bottomrule| vêm do pacote {\tt booktabs}
        e dão bom aspecto à tabela.

        Criar células que abrangem múltiplas colunas ou linhas geram uma tabela mais complicada.
        Nesse caso, pode-se lançar mão dos comandos \verb|\multirow{}{}{}| e \verb|\multicolumn{}{}{}|,
        que recebem a quantidade de colunas/linhas a serem mescladas, 
        o tipo de centralização da célula resultante,
        e o valor da célula resultante.
        Por exemplo, o código
        \begin{verbatim}
        \begin{table}[H]\centering
            \begin{tabular}{c c c}
                \toprule
                \multirow{2}{*}{$n$} & \multicolumn{2}{c}{$I_1$} \\
                \cmidrule{2-3}
                    & teórico & real   \\\midrule\midrule
                 40 &  14256  & 14256  \\
                 50 &  22316  & 22316  \\
                 60 &  32176  & 32176  \\
                 70 &  43836  & 43836  \\
                 80 &  57296  & 57296  \\
                 90 &  72556  & 72556  \\
                100 &  89616  & 89616  \\\bottomrule
            \end{tabular}
        \end{table}
        \end{verbatim}

        gera a tabela
        \begin{table}[H]\centering
            \begin{tabular}{c c c}
                \toprule
                \multirow{2}{*}{$n$} & \multicolumn{2}{c}{$I_1$} \\
                \cmidrule{2-3}
                    & teórico & real   \\\midrule\midrule
                 40 &  14256  & 14256  \\
                 50 &  22316  & 22316  \\
                 60 &  32176  & 32176  \\
                 70 &  43836  & 43836  \\
                 80 &  57296  & 57296  \\
                 90 &  72556  & 72556  \\
                100 &  89616  & 89616  \\\bottomrule
            \end{tabular}
        \end{table}

        Note ainda o uso do comando \verb|\cmidrule{}|, uma versão mais abrangente do \verb|\midrule|;
        recebe como argumento o números das colunas pelas quais a linha horizontal deve passar.
        No exemplo, ela atravessa a segunda e terceira coluna, evitando o $n$.
        Para entender melhor a criação de tabelas e/ou os comandos do {\tt booktab}, visite
        \href{https://www.overleaf.com/learn/latex/Tables}{o Overleaf} ou a 
        \href{https://linorg.usp.br/CTAN/macros/latex/contrib/booktabs/booktabs.pdf}{documentação do pacote}.
       % seção postiça 2  
    \begin{thebibliography}{5}
    \bibitem{tipler} 
        Paul A. Tipler e Gene Mosca.
        \textit{Física Volume 2, 5\textordfeminine Edição}. 
        LTC--Livros Técnicos e Científicos Editora S.A., Rio de Janeiro, 2006. 
    
    \bibitem{taylor}
        John R. Taylor. 
        \textit{An Introduction to Error Analysis, Second Edition}. 
        University Science Books, Sausalito (CA), 1997. 

    \bibitem{britannica}
        Britannica, The Editors of Encyclopaedia. 
        "servomechanism".
        Encyclopedia Britannica, 14 May. 2013, 
        \url{https://www.britannica.com/technology/servomechanism}. 
        Acessado 15 de janeiro de 2023.
\end{thebibliography}   % bibliografia
\end{document}
