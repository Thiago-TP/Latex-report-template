\usepackage[T1]{fontenc}    % suporte para letras com acento e hifenização 
                            % (https://tex.stackexchange.com/questions/664/why-should-i-use-usepackaget1fontenc)
                            
% definição da cor tema (mais cores em latexcolor.com)
\usepackage{tikz}
    \definecolor{themecolor}{rgb}{1.0, 0.65, 0.0}
    
\usepackage[margin=2.5cm]{geometry} % margem do texto uniforme e igual a 2.5cm
\usepackage[brazil]{babel}          % troca 'Summary' por 'Sumário', 'Figure' por 'Figura', 'Table' por 'Tabela', etc.
\usepackage{charter}                % coloca a fonte do relatório como charter
\usepackage{helvet}                 % estiliza a fonte sans serif 
                                    % (https://linorg.usp.br/CTAN/macros/latex/required/psnfss/psnfss2e.pdf)
\usepackage{multirow}               % fornece o comando \multirow{}{}{} para mesclar células numa mesma linha da tabela
\usepackage{booktabs}               % disponibiliza comandos que deixam a tabela mais bonita 
                                    % (https://linorg.usp.br/CTAN/macros/latex/contrib/booktabs/booktabs.pdf) 
\usepackage{graphicx, float}        % pacotes para inserção de figuras
    \graphicspath{{figures/}}       % pasta figures se torna o lugar de procura padrão do \includegraphics
\usepackage{amsmath}                % pacote básico para suporte a equações e matemática em geral 
                                    % (https://linorg.usp.br/CTAN/macros/latex/required/amsmath/amsldoc.pdf)
\usepackage{listings}                       % pacote para incluir trechos de código no latex
% crédito: https://nasa.github.io/nasa-latex-docs/html/examples/listing.html
    % Define a custom color
    \definecolor{codegreen} {rgb}{0.00, 0.60, 0.00}
    \definecolor{codegray}  {rgb}{0.50, 0.50, 0.50}
    \definecolor{codepurple}{rgb}{0.58, 0.00, 0.82}
    \definecolor{backcolour}{rgb}{0.95, 0.95, 0.92}

    % Define a custom style
    \lstdefinestyle{myStyle}{
        backgroundcolor     = \color{backcolour},   
        commentstyle        = \color{codegreen},
        keywordstyle        = \color{magenta},
        numberstyle         = \sffamily\tiny\color{codegray},
        stringstyle         = \color{codepurple},
        basicstyle          = \ttfamily\footnotesize,
        breakatwhitespace   = false,         
        breaklines          = true,  
        keepspaces          = true,                 
        numbers             = left,
        numbersep           = 5pt,                  
        showspaces          = false,                
        showstringspaces    = false,
        showtabs            = false,                  
        tabsize             = 2
    }
    %
    
    \lstset{
        style=myStyle,
        inputpath={listings/},  
        extendedchars=true, literate=   % suporte para caracteres fora do ASCII
        {á}{{\'a}}1 {é}{{\'e}}1 {í}{{\'i}}1 {ó}{{\'o}}1 {ú}{{\'u}}1
        {Á}{{\'A}}1 {É}{{\'E}}1 {Í}{{\'I}}1 {Ó}{{\'O}}1 {Ú}{{\'U}}1
        {à}{{\`a}}1 {è}{{\`e}}1 {ì}{{\`i}}1 {ò}{{\`o}}1 {ù}{{\`u}}1
        {À}{{\`A}}1 {È}{{\'E}}1 {Ì}{{\`I}}1 {Ò}{{\`O}}1 {Ù}{{\`U}}1
        {ä}{{\"a}}1 {ë}{{\"e}}1 {ï}{{\"i}}1 {ö}{{\"o}}1 {ü}{{\"u}}1
        {Ä}{{\"A}}1 {Ë}{{\"E}}1 {Ï}{{\"I}}1 {Ö}{{\"O}}1 {Ü}{{\"U}}1
        {â}{{\^a}}1 {ê}{{\^e}}1 {î}{{\^i}}1 {ô}{{\^o}}1 {û}{{\^u}}1
        {Â}{{\^A}}1 {Ê}{{\^E}}1 {Î}{{\^I}}1 {Ô}{{\^O}}1 {Û}{{\^U}}1
        {ã}{{\~a}}1 {ẽ}{{\~e}}1 {ĩ}{{\~i}}1 {õ}{{\~o}}1 {ũ}{{\~u}}1
        {Ã}{{\~A}}1 {Ẽ}{{\~E}}1 {Ĩ}{{\~I}}1 {Õ}{{\~O}}1 {Ũ}{{\~U}}1
        {œ}{{\oe}}1 {Œ}{{\OE}}1 {æ}{{\ae}}1 {Æ}{{\AE}}1 {ß}{{\ss}}1
        {ű}{{\H{u}}}1 {Ű}{{\H{U}}}1 {ő}{{\H{o}}}1 {Ő}{{\H{O}}}1
        {ç}{{\c c}}1 {Ç}{{\c C}}1 {ø}{{\o}}1 {å}{{\r a}}1 {Å}{{\r A}}1
        {€}{{\euro}}1 {£}{{\pounds}}1 {«}{{\guillemotleft}}1
        {»}{{\guillemotright}}1 {ñ}{{\~n}}1 {Ñ}{{\~N}}1 {¿}{{?`}}1 {¡}{{!`}}1 
    }          % formatação de códigos python e matlab
                                    % (http://users.ece.utexas.edu/~garg/dist/listings.pdf)

% estiliza texto das legendas
\usepackage{caption, subcaption}
    \captionsetup{
        labelfont={sf, bf},         % 'Figura X:' em negrito e sans serif
        format={hang},              % quebra de linha no texto da legenda é identado
        textfont=it                 % texto da legenda em itálico
    }

% estilização dos títulos das seções, subseções, e subsubseções
\usepackage{titlesec} 
    \titleformat*{\section}{\large\bfseries\sffamily}   % fonte grande, negrito, sans serif
    \titleformat*{\subsection}{\bfseries\sffamily}      % negrito, sans serif
    \titleformat*{\subsubsection}{\sffamily}            % sans serif

% escreve informações que devem aparecer no topo e base da página
\usepackage{fancyhdr, lastpage}
    \setlength{\headheight}{15pt}                       % espaçamento de segurança entre o cabeçalho e o texto
    \fancyhead{}
    \fancyhead[r]{Código do curso -- Prof. Dr. Nome do professor}
    \fancyhead[l]{\rightmark}                           % subseção atual
    
    \fancyfoot{} 
    \fancyfoot[r]{\thepage \ de \pageref*{LastPage}}    % página X de Y
    \fancyfoot[l]{\leftmark}                            % seção atual
    
    \renewcommand{\headrulewidth}{0.00mm}               % sem linha no topo da página (tire do 0 para criá-la)
    \renewcommand{\footrulewidth}{0.25mm}               % com linha no fundo da página (ponha em 0 para tirá-la)
    \pagestyle{fancy}                                   % moldura acima = padrão do documento (exceto página do título)

% coloração dos links
\usepackage{hyperref}
    \hypersetup{
        colorlinks,
        linkcolor   = themecolor!80!black,  % mistura a cor tema com um pouco de preto
        urlcolor    = themecolor!80!black
    }


% comando para a criação da figura no título
\newcommand{\cover}[1]{ 
    \begin{center}
        \includegraphics[width=5cm]{#1}
    \end{center}
}

% comando para destacar texto (caixa colorida)
\newcommand{\highlightbox}[1]{\noindent
    \colorbox{themecolor!25}{
        \begin{minipage}{\linewidth}
            #1
        \end{minipage}
    }
}
